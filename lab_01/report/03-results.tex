\chapter{Результаты расчетов для выборки из индивидуального варианта}

Результаты расчетов, выполненных с помощью реализованной программы для выборки из индивидуального варианта №1, представлены в формулах \ref{eq:res1}, \ref{eq:res2}, \ref{eq:res3}, \ref{eq:res4}, \ref{eq:res5}, \ref{eq:res6}.

\begin{equation}\label{eq:res1}
	M_max = 1.4
\end{equation}

\begin{equation}\label{eq:res2}
	M_min = -4.11
\end{equation}

\begin{equation}\label{eq:res3}
	R = 5.51
\end{equation}

\begin{equation}\label{eq:res4}
	\hat{\mu}(\vec{x}) = -1.6046
\end{equation}

\begin{equation}\label{eq:res5}
	S^2(\vec{x}) = 1.0341
\end{equation}

\begin{equation}\label{eq:res6}
	m = 8
\end{equation}

Результат группировки значений выборки в $m = 8$ интервалов представлен на рисунке \ref{img:res}.

\includeimage{res}{f}{h!}{0.9\textwidth}{Результат группировки значений выборки}

\clearpage

На рисунке \ref{img:hist} представлены гистограмма и график функции плотности распредления нормальной случайной величины с математическим ожиданием $\hat{\mu}$ и дисперсией $S^2$.

\includeimage{hist}{f}{h!}{0.9\textwidth}{Гистограмма и график функции плотности распредления нормальной случайной величины}

\clearpage

На рисунке \ref{img:empiric} представлены графики эмпирической функции распределения и функции распределения нормальной случайной величины с математическим ожиданием $\hat{\mu}$ и дисперсией $S^2$.

\includeimage{empiric}{f}{h!}{0.9\textwidth}{Графики эмпирической функции распределения и функции распределения нормальной случайной величины}