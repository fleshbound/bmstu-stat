\chapter{Теоретические сведения}

\section{Формулы}

Максимальное значение:

\begin{equation}\label{eq:mmax}
	M_{max} = x_{(1)} = \max\limits_{x_i \in \vec{x}} x_i.
\end{equation}

Минимальное значение:

\begin{equation}\label{eq:mmin}
	M_{min} = x_{(n)} = \min\limits_{x_i \in \vec{x}} x_i.
\end{equation}

Размах выборки:

\begin{equation}\label{eq:r}
	R = M_{max} - M_{min}
\end{equation}

Оценка математического ожидания (выборочное среднее):

\begin{equation}\label{eq:mu}
	\hat{\mu}(\vec{x}) = \frac{1}{n} \sum\limits_{i = 1}^{n} x_i
\end{equation}

Исправленная оценка дисперсии (исправленная выборочная дисперсия):

\begin{equation}\label{eq:s_quad}
	S^2(\vec{x}) = \frac{1}{n - 1} \sum\limits_{i = 1}^{n} (x_i - \hat{\mu}(\vec{x}))^2
\end{equation}

\section{Определения}

Пусть $\vec{x} = (x_1, \dots, x_n)$ --- реализация случайной выборки $\vec{X}=(X_1, \dots, X_n)$ из генеральной совокупности $X$.
Пусть также $m \in \mathbb{N}$ --- количество интервалов, а отрезок $J = [ x_{(1)}; \; x_{(n)} ]$ разбивают на $m$ равновеликих промежутков длины

\begin{equation}\label{eq:delta}
	\Delta = \frac{x_{(n)} -  x_{(1)}}{m}.
\end{equation}

Интервалы определяются равенствами \ref{eq:1}, \ref{eq:2}:

\begin{equation}\label{eq:1}
	J_i = [x_{(1)} + (i - 1) \Delta; \; x_{(i)} + i \Delta), \quad i = 1, \dots, m-1;
\end{equation}

\begin{equation}\label{eq:2}
	J_m = [x_{(1)} + (m - 1) \Delta; \; x_{(i)} + m \Delta).
\end{equation}

\textit{Интервальным статистическим рядом} называют таблицу \ref{eq:table}

\begin{equation}\label{eq:table}
	\begin{array}{|c|c|c|}
		\hline
		J_1 & \text{\dots} & J_m \\
		\hline
		n_1 & \text{\dots} & n_m \\
		\hline
	\end{array}
\end{equation}

где $n_i$ --- число элементов в реализации $\vec{x} = (x_1, \dots, x_n)$, попавших в промежуток $J_i$, $i = 1, \dots, m$

Предположим, что для данной реализации $\vec{x}$ построен интервальный статистический ряд.

\textit{Эмпирической плотностью распределения} называют функцию:

\begin{equation}\label{eq:empiric_density}
	f_n(t) = \begin{cases}
		\frac{n_i}{n \cdot \Delta}, & t \in J_i, \quad i = 1, \dots, m, \\
		0, & t \notin J.
	\end{cases}
\end{equation}

График функции $f_n(t)$ называют \textit{гистограммой}.

Пусть $\vec{x} = (x_1, \dots, x_n)$ --- реализация случайной выборки $\vec{X} = (X_1, \dots, X_n)$ из генеральной совокупности $X$.
Обозначим с помощью $l(t, \vec{x})$ число элементов $\vec{x}$, которые меньше $t$ ($t \in \mathbb{R}$).

\textit{Эмпирической функцией} распределения называют отображение 

\begin{equation}
	F_n: \mathbb{R} \to \mathbb{R},
\end{equation}

заданное формулой 

\begin{equation}\label{eq:empiric_f}
	F_n(t) = \frac{l(t, \vec{x})}{n}.
\end{equation}