\chapter{Результаты расчетов для выборки из индивидуального варианта}

На листинге \ref{lst:results.txt} представлены результаты расчетов для выборки из индивидуального варианта №1.

\includelistingpretty{results.txt}{Octave}{Результаты расчетов для выборки из индивидуального варианта}

На рисунках \ref{img:mu_1_120_lines}, \ref{img:sigma_1_120_lines}, \ref{img:mu_10_120_lines}, \ref{img:sigma_10_120_lines} представлены графики для выборки из индивидуального варианта (при построении графиков $\gamma = 0.9$).

\includeimage{mu_1_120_lines}{f}{h!}{0.8\textwidth}{Прямая $y=\hat \mu (\vec x_N)$ и графики функций $y=\hat \mu (\vec x_n)$, $y=\underline \mu (\vec x_n)$ и $y=\overline \mu (\vec x_n)$ как функций объема $n$ выборки, где $n$ изменяется от $1$ до $N$}

\includeimage{sigma_1_120_lines}{f}{h!}{0.8\textwidth}{Прямая \mbox{$z=S^2 (\vec x_N)$} и графики функций $z=S^2 (\vec x_n)$, $z=\underline \sigma^2 (\vec x_n)$ и \mbox{$z=\overline \sigma^2 (\vec x_n)$} как функций объема $n$ выборки, где $n$ изменяется от $1$ до $N$}

\includeimage{mu_10_120_lines}{f}{h!}{0.8\textwidth}{Прямая $y=\hat \mu (\vec x_N)$ и графики функций $y=\hat \mu (\vec x_n)$, $y=\underline \mu (\vec x_n)$ и $y=\overline \mu (\vec x_n)$ как функций объема $n$ выборки, где $n$ изменяется от $10$ до $N$}

\includeimage{sigma_10_120_lines}{f}{h!}{0.8\textwidth}{Прямая \mbox{$z=S^2 (\vec x_N)$} и графики функций $z=S^2 (\vec x_n)$, $z=\underline \sigma^2 (\vec x_n)$ и \mbox{$z=\overline \sigma^2 (\vec x_n)$} как функций объема $n$ выборки, где $n$ изменяется от $10$ до $N$}