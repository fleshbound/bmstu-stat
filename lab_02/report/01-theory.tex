\chapter{Теоретические сведения}

\section{Определения}

Рассматривается вторая основная задача математической статистики.

Дано: случайная величина $X$, закон распределения которой известен с точностью до неизвестного параметра $\theta$.

Требуется: оценить значение $\theta$.

\textit{Интервальной оценкой уровня $\gamma$ (с коэффициентом доверия $\gamma$) параметра $\theta$} называют пару статистик $\ubar{\theta}(\vec{X})$ и $\bar{\theta}(\vec{X})$ таких, что

\begin{equation}
	P\{ \, \ubar{\theta}(\vec{X}) < \theta < \bar{\theta}(\vec{X}) \, \} = \gamma.
\end{equation}

\textit{Доверительным интервалом уровня $\gamma$ ($\gamma$--доверительным интервалом)} называется интервал

\begin{equation}
	(\ubar{\theta}(\vec{x}), \bar{\theta}(\vec{x})),
\end{equation}

\noindent отвечающий выборочным значениям $\ubar{\theta}(\vec{X})$ и $\bar{\theta}(\vec{X})$.

Статистика $T(\vec{X}, \theta)$ называется \textit{центральной}, если закон ее распределения не зависит от $\theta$.

\section{Формулы}

Будем предполагать, что

\begin{enumerate}
	\item $X$ --- случайная величина, закон распределения которой зависит от неизвестного параметра $\theta$;
	\item $\gamma \in (0; 1)$ --- выбранный уровень доверия;
	\item $T(\vec{X}, \theta)$ --- центральная статистика;
	\item $T$ является непрерывной;
	\item функция $F_T(t)$ распределения статистики $T$ является монотонно возрастающей на множестве $\{t \in \mathbb{R}: \, 0 < F_T(t) < 1\}$;
	\item $T$ является монотонно возрастающей функцией параметра $\theta$;
\end{enumerate}

Выберем значения $\alpha_1, \, \alpha_2 \in \mathbb{R}$ так, чтобы $\alpha_1 + \alpha_2 = 1 - \gamma$, при этом $\alpha_1 \geq~0, \, \alpha_2 \geq 0$.
Пусть $q_{\alpha_1}$, $q_{1 - \alpha_2}$ --- квантили соответствующих уровней случайной величины $T$.
Тогда в соответствии с предположениями 1--6 получаем: 

\begin{equation}
	\gamma = P\{ \, T^{-1}(\vec{X}, q_{\alpha_1}) < \theta < T^{-1}(\vec{X}, q_{1 - \alpha_2}) \, \}.
\end{equation}

Пусть $X \sim N(\theta, \sigma^2)$, где $\theta$ --- неизв., $\sigma^2$ --- изв.

Тогда для построения $\gamma$--доверительного интервала для математического ожидания $\theta$ используется следующая центральная статистика и ее закон распределения:

\begin{equation}
	T(\vec{X}, \theta) = \frac{\theta - \bar{X}}{\sigma}\sqrt{n} \sim N(0, 1).
\end{equation}

В результате построения $\gamma$--доверительного интервала для математического ожидания $\theta$ получаются формулы для вычисления границ этого интервала ($U_{\alpha_1}, U_{1 - \alpha_2}$ --- квантили соответствующих уровней):

\begin{equation}\label{eq:theta_low}
	\ubar{\theta}(\vec{X}) = \bar{X} + \frac{\sigma U_{\alpha_1}}{\sqrt{n}},
\end{equation}

\begin{equation}\label{eq:theta_high}
	\bar{\theta}(\vec{X}) = \bar{X} + \frac{\sigma U_{1 - \alpha_2}}{\sqrt{n}}.
\end{equation}

При $\alpha_1 = \alpha_2$ формулы \ref{eq:theta_low}, \ref{eq:theta_high} принимают вид:

\begin{equation}
	\ubar{\theta}(\vec{X}) = \bar{X} - \frac{\sigma U_{ \frac{1 + \gamma}{2} } }{ \sqrt{n} },
\end{equation}

\begin{equation}
	\bar{\theta}(\vec{X}) = \bar{X} + \frac{\sigma U_{ \frac{1 + \gamma}{2} } }{ \sqrt{n} }.
\end{equation}

В случае, если $X \sim N(\theta, \sigma^2)$, $\theta$ --- неизв., $\sigma^2$ --- неизв., для построения $\gamma$--доверительного интервала для математического ожидания $\theta$ используется следующая центральная статистика и ее закон распределения:

\begin{equation}
	T(\vec{X}, \theta) = \frac{\theta - \bar{X}}{S(\vec{X})}\sqrt{n} \sim St(n - 1),
\end{equation}

где $S^2(\vec{X}) = \frac{1}{n - 1} \sum\limits_{i - 1}^{n} (X_i - \bar{X})^2$ --- исправленная выборочная дисперсия.

Тогда по аналогии с предыдущим случаем получаем в результате построения $\gamma$--доверительного интервала для математического ожидания $\theta$ формулы для вычисления границ этого интервала ($t_{\alpha_1}, t_{1 - \alpha_2}$ --- квантили соответствующих уровней):

\begin{equation}\label{eq:theta_low1}
	\ubar{\theta}(\vec{X}) = \bar{X} + \frac{S(\vec{X}) t_{\alpha_1}}{\sqrt{n}},
\end{equation}

\begin{equation}\label{eq:theta_high1}
	\bar{\theta}(\vec{X}) = \bar{X} + \frac{S(\vec{X}) t_{1 - \alpha_2}}{\sqrt{n}}.
\end{equation}

При $\alpha_1 = \alpha_2$ формулы \ref{eq:theta_low1}, \ref{eq:theta_high1} принимают вид:

\begin{equation}
	\ubar{\theta}(\vec{X}) = \bar{X} - \frac{S(\vec{X}) t_{ \frac{1 + \gamma}{2} } }{ \sqrt{n} },
\end{equation}

\begin{equation}
	\bar{\theta}(\vec{X}) = \bar{X} + \frac{S(\vec{X}) t_{ \frac{1 + \gamma}{2} } }{ \sqrt{n} }.
\end{equation}

В случае, если $X \sim N(m, \theta)$, $m$ --- неизв., $\theta$ --- неизв., для построения $\gamma$--доверительного интервала для дисперсии $\theta$ используется следующая центральная статистика и ее закон распределения:

\begin{equation}
	T(\vec{X}, \theta) = \frac{S^2(\vec{X})}{\theta})(n - 1) \sim \chi^2(n - 1).
\end{equation}

По аналогии с предыдущим случаем получаем в результате построения $\gamma$--доверительного интервала для дисперсии $\sigma^2$ формулы для вычисления границ этого интервала ($h_{\alpha_1}, h_{1 - \alpha_2}$ --- квантили соответствующих уровней):

\begin{equation}
	\ubar{\theta}(\vec{X}) = \frac{(n - 1) S^2(\vec{X})}{h_{1 - \alpha_2}},
\end{equation}

\begin{equation}
	\bar{\theta}(\vec{X}) = \frac{(n - 1) S^2(\vec{X})}{h_{\alpha_1}}.
\end{equation}

При $\alpha_1 = \alpha_2: \quad h_{\alpha_1} = h_{\frac{1 - \gamma}{2}}$, $h_{1 - \alpha_2} = h_{\frac{1 + \gamma}{2}}$.